\documentclass[12pt]{article}
\usepackage[margin=0.2in]{geometry}


%-------Packages---------
\usepackage{amssymb,amsfonts, fullpage}
\usepackage{amsmath,amssymb,amsthm,fullpage,verbatim}
\usepackage[all,arc]{xy}
\usepackage{enumerate}
\usepackage{mathrsfs}
\usepackage{graphicx}
\usepackage{mathtools}
\usepackage{pst-node}
\usepackage{tikz-cd} 
\usepackage{tikz} 
\usepackage{bussproofs}
\usepackage{dsfont}
\usepackage[symbol]{footmisc}
\usepackage{hyperref}
\usepackage{circuitikz}
\usetikzlibrary{calc}

\let\oldemptyset\emptyset
\let\emptyset\varnothing

% Commands
\newcommand{\defeq}{\mathrel{\mathop:}=}
\newcommand{\bigslant}[2]{{\raisebox{.2em}{$#1$}\left/\raisebox{-.2em}{$#2$}\right.}}

%--------Theorem Environments--------
%theoremstyle{plain} --- default
\newtheorem{thm}{Theorem}[section]
\newtheorem{cor}[thm]{Corollary}
\newtheorem{prop}[thm]{Proposition}
\newtheorem{lem}[thm]{Lemma}
\newtheorem{lemm}[thm]{Lemming}
\newtheorem{HW}[thm]{HW}
\newtheorem{DO}[thm]{DO}
\newtheorem{CH}[thm]{CH}
\newtheorem{conj}[thm]{Conjecture}
\newtheorem{quest}[thm]{Question}
\newtheorem{axiom}{Axiom}

\newcommand{\vect}[1]{\Vec{\textbf{#1}}}

\theoremstyle{definition}
\newtheorem{defn}[thm]{Definition}
\newtheorem{defns}[thm]{Definitions}
\newtheorem{con}[thm]{Construction}
\newtheorem{exmp}[thm]{Example}
\newtheorem{sexmp}[thm]{Sub-example}
\newtheorem{exmps}[thm]{Examples}
\newtheorem{notn}[thm]{Notation}
\newtheorem{notns}[thm]{Notations}
\newtheorem{addm}[thm]{Addendum}
\newtheorem{exer}[thm]{Exercise}
\newtheorem{alg}[thm]{Algorithm}
\newtheorem{claim}[thm]{Claim}
\newtheorem{prob}[thm]{Problem}

%\theoremstyle{remark}
\newtheorem{rem}[thm]{Remark}
\newtheorem{rems}[thm]{Remarks}
\newtheorem{warn}[thm]{Warning}
\newtheorem{sch}[thm]{Scholium}

\makeatletter
\let\c@equation\c@thm
\makeatother
%\numberwithin{equation}{section}
\setlength\parindent{0pt}

\bibliographystyle{plain}

\makeatletter
\def\class#1{\gdef\@class{#1}}
\def\@maketitle{%
  \newpage \null  \vskip 2em%
  \begin{flushright}%
  \let\footnote\thanks
    \vskip 1.5em%
    {\large  \lineskip .5em%
      \begin{tabular}[t]{r@{}}%
        \@author    
     \end{tabular}\par 
      \begin{tabular}[t]{r@{}}%
      {\sffamily \@class}   
     \end{tabular}\par 
  \par}%
    \vskip .5em%
   {\small\@date}\par\bigskip%
    {\LARGE\scshape \@title \par\bigskip\hrule}%
  \end{flushright}%
  \par   \vskip 1.5em}
\makeatother

\title{HW 4F: Linear Systems, Linear Combinations, \& Matrix Equations}
\class{Fall 2024} 
\author{CSCI 2820: Linear Algebra \& CS applications }
\date{due date: Friday, September 20 at 10:00 am}

\begin{document}
\maketitle
\renewcommand{\thefootnote}{\fnsymbol{footnote}}

The goal of this assignment is to convince you that the following 3 problems are actually all just different ways of framing the same problem:
\begin{enumerate}
    \item Solving a system of linear number equations
    \item Solving a linear vector equation -- i.e. determining whether one vector can be written as a linear combination of other vectors
    \item Solving a matrix equation -- i.e. given a matrix $A$ and a vector $\vect{b}$, determine if there is a vector $\vect{x}$ such that $A\vect{x} = \vect{b}$
\end{enumerate}

\newpage
\section{Understanding Linear Systems}
Lay \& McDonald, Chapter 1.3 exercises 5-10. \\

In exercises 5, 6, 9, and 10, write the system as a matrix equation $A \vect{x} = \vect{b}$ as well. 

{\color{red}

\begin{enumerate}
    \item[5.] 
    The vector equation:
    \[
    x_1 \begin{bmatrix} 6 \\ -1 \\ 5 \end{bmatrix} + x_2 \begin{bmatrix} -3 \\ 4 \\ 0 \end{bmatrix} = \begin{bmatrix} 1 \\ -7 \\ -5 \end{bmatrix}
    \]
    The matrix equation $A \vect{x} = \vect{b}$:
    \[
    \begin{bmatrix} 
    6 & -3 \\
    -1 & 4 \\
    5 & 0 
    \end{bmatrix}
    \begin{bmatrix} 
    x_1 \\ x_2 
    \end{bmatrix} = 
    \begin{bmatrix} 
    1 \\ -7 \\ -5 
    \end{bmatrix}
    \]
    
    \item[6.] 
    The vector equation:
    \[
    x_1 \begin{bmatrix} -2 \\ 3 \end{bmatrix} + x_2 \begin{bmatrix} 8 \\ 5 \end{bmatrix} + x_3 \begin{bmatrix} 1 \\ -6 \end{bmatrix} = \begin{bmatrix} 0 \\ 0 \end{bmatrix}
    \]
    The matrix equation $A \vect{x} = \vect{b}$:
    \[
    \begin{bmatrix} 
    -2 & 8 & 1 \\
    3 & 5 & -6
    \end{bmatrix}
    \begin{bmatrix} 
    x_1 \\ x_2 \\ x_3
    \end{bmatrix} = 
    \begin{bmatrix} 
    0 \\ 0 
    \end{bmatrix}
    \]
    
    \item[9.] 
    The vector equation:
    \[
    x_1 \begin{bmatrix} 0 \\ 4 \\ -1 \end{bmatrix} + x_2 \begin{bmatrix} 1 \\ 6 \\ 3 \end{bmatrix} + x_3 \begin{bmatrix} 5 \\ -1 \\ -8 \end{bmatrix} = \begin{bmatrix} 0 \\ 0 \\ 0 \end{bmatrix}
    \]
    The matrix equation $A \vect{x} = \vect{b}$:
    \[
    \begin{bmatrix} 
    0 & 1 & 5 \\
    4 & 6 & -1 \\
    -1 & 3 & -8
    \end{bmatrix}
    \begin{bmatrix} 
    x_1 \\ x_2 \\ x_3
    \end{bmatrix} = 
    \begin{bmatrix} 
    0 \\ 0 \\ 0
    \end{bmatrix}
    \]
    
    \item[10.] 
    The vector equation:
    \[
    x_1 \begin{bmatrix} 4 \\ 1 \\ 8 \end{bmatrix} + x_2 \begin{bmatrix} 1 \\ -7 \\ 6 \end{bmatrix} + x_3 \begin{bmatrix} 3 \\ -2 \\ -5 \end{bmatrix} = \begin{bmatrix} 9 \\ 2 \\ 15 \end{bmatrix}
    \]
    The matrix equation $A \vect{x} = \vect{b}$:
    \[
    \begin{bmatrix} 
    4 & 1 & 3 \\
    1 & -7 & -2 \\
    8 & 6 & -5
    \end{bmatrix}
    \begin{bmatrix} 
    x_1 \\ x_2 \\ x_3
    \end{bmatrix} = 
    \begin{bmatrix} 
    9 \\ 2 \\ 15
    \end{bmatrix}
    \]
\end{enumerate}

}

\newpage
\section{Solving Linear Systems}

\begin{enumerate}
    \item[a.] Lay \& McDonald, Chapter 1.1 exercises 3-4, 11-12
    \item[b.] Lay \& McDonald, Chapter 1.3 exercise 26
\end{enumerate}

{\color{red}
\begin{enumerate}
    \item[3.] 
    Find the point $(x_1, x_2)$ that lies on the lines: 
    \[
    \begin{aligned}
        x_1 + 5x_2 &= 7 \\
        x_1 - 2x_2 &= -2
    \end{aligned}
    \]
    **Solution:** Solving this system, we find \( x_1 = 3 \) and \( x_2 = 4/5 \). So the point is \( (3, 4/5) \).
    
    \item[4.] 
    Find the point of intersection of the lines: 
    \[
    \begin{aligned}
        x_1 - 5x_2 &= 1 \\
        3x_1 - 7x_2 &= 5
    \end{aligned}
    \]
    **Solution:** Solving this system, we find \( x_1 = 6 \) and \( x_2 = 1 \). So the point of intersection is \( (6, 1) \).
    
    \item[11.] 
    Solve the system:
    \[
    \begin{aligned}
        x_2 + 4x_3 &= -5 \\
        x_1 + 3x_2 + 5x_3 &= -2 \\
        3x_1 + 7x_2 + 7x_3 &= 6
    \end{aligned}
    \]
    **Solution:** Using Gaussian elimination, we find \( x_1 = 2 \), \( x_2 = 1 \), and \( x_3 = -2 \).
    
    \item[12.] 
    Solve the system:
    \[
    \begin{aligned}
        x_1 - 3x_2 + 4x_3 &= -4 \\
        3x_1 - 7x_2 + 7x_3 &= -8 \\
        -4x_1 + 6x_2 - x_3 &= 7
    \end{aligned}
    \]
    **Solution:** After row reduction, the solution is \( x_1 = -1 \), \( x_2 = 2 \), and \( x_3 = -3 \).
    
    \item[26.] 
    Let 
    \[
    A = \begin{bmatrix} 2 & 0 & 6 \\ -1 & 8 & 5 \\ 1 & -2 & 1 \end{bmatrix}, \quad \mathbf{b} = \begin{bmatrix} 10 \\ 3 \\ 3 \end{bmatrix}
    \]
    and let \( W \) be the set of all linear combinations of the columns of \( A \).

    \begin{itemize}
        \item[a.] Is \( \mathbf{b} \in W \)? 
        **Solution:** No, \( \mathbf{b} \) is not in \( W \) because the system \( A \mathbf{x} = \mathbf{b} \) does not have a solution.
        
        \item[b.] Show that the third column of \( A \) is in \( W \). 
        **Solution:** The third column is in \( W \) because it is part of the span of the columns of \( A \).
    \end{itemize}
\end{enumerate}

}

\newpage
\section{Matrix Equations \& Column Space}

Lay \& McDonald, Chapter 1.4 exercises 15-20, 25, 35-36

{\color{red}
\begin{enumerate}
    \item[15.] 
    Let 
    \[
    A = \begin{bmatrix} 2 & -1 \\ -6 & 3 \end{bmatrix}, \quad \mathbf{b} = \begin{bmatrix} b_1 \\ b_2 \end{bmatrix}
    \]
    **Solution:** The determinant of \( A \) is zero, so the system has solutions only when \( b_1 \) and \( b_2 \) are dependent. If \( b_1 = -2b_2 \), then there is a solution.
    
    \item[16.]
    Repeat Exercise 15 with 
    \[
    A = \begin{bmatrix} 1 & -3 & -4 \\ -3 & 2 & 6 \\ 5 & -1 & -8 \end{bmatrix}, \quad \mathbf{b} = \begin{bmatrix} b_1 \\ b_2 \\ b_3 \end{bmatrix}
    \]
    **Solution:** Using Gaussian elimination, we find that a solution exists only if \( b_1 + b_2 - 2b_3 = 0 \).

    \item[17.]
    How many rows of \( A \) contain a pivot position? Does the equation \( A\mathbf{x} = \mathbf{b} \) have a solution for each \( \mathbf{b} \in \mathbb{R}^4 \)?
    **Solution:** Perform Gaussian elimination to determine the number of pivots. If the number of pivots is equal to the number of rows, then there is a solution for each \( \mathbf{b} \in \mathbb{R}^4 \).
    
    \item[18.]
    Do the columns of \( B \) span \( \mathbb{R}^4 \)? Does the equation \( B\mathbf{x} = \mathbf{y} \) have a solution for each \( \mathbf{y} \in \mathbb{R}^4 \)?
    **Solution:** Perform Gaussian elimination on \( B \). If there are 4 pivot positions, then the columns span \( \mathbb{R}^4 \), and the system has a solution for each \( \mathbf{y} \).
    
    \item[19.] 
    Can each vector in \( \mathbb{R}^4 \) be written as a linear combination of the columns of the matrix \( A \)? Do the columns of \( A \) span \( \mathbb{R}^4 \)?
    **Solution:** If \( A \) has full rank, its columns span \( \mathbb{R}^4 \). Perform Gaussian elimination to verify.

    \item[20.] 
    Can every vector in \( \mathbb{R}^4 \) be written as a linear combination of the columns of the matrix \( B \)? Do the columns of \( B \) span \( \mathbb{R}^3 \)?
    **Solution:** If the rank of \( B \) is 3, then the columns span \( \mathbb{R}^3 \) but not \( \mathbb{R}^4 \).
    
    \item[25.]
    Given 
    \[
    \begin{bmatrix} 4 & -3 & 1 \\ 5 & -2 & 5 \\ -6 & 2 & -3 \end{bmatrix} \begin{bmatrix} -3 \\ -1 \\ 2 \end{bmatrix} = \begin{bmatrix} -7 \\ -3 \\ 10 \end{bmatrix}
    \]
    Find \( c_1, c_2, c_3 \) such that 
    \[
    \begin{bmatrix} -7 \\ -3 \\ 10 \end{bmatrix} = c_1 \begin{bmatrix} 4 \\ 5 \\ -6 \end{bmatrix} + c_2 \begin{bmatrix} -3 \\ -2 \\ 2 \end{bmatrix} + c_3 \begin{bmatrix} 1 \\ 5 \\ -3 \end{bmatrix}
    \]
    **Solution:** Solving this system, we find \( c_1 = 1 \), \( c_2 = 2 \), and \( c_3 = -1 \).

    \item[35.] 
    Let \( A \) be a \( 3 \times 4 \) matrix, let \( \mathbf{y}_1 \) and \( \mathbf{y}_2 \) be vectors in \( \mathbb{R}^3 \), and let \( \mathbf{w} = \mathbf{y}_1 + \mathbf{y}_2 \). Suppose \( \mathbf{y}_1 = A\mathbf{x}_1 \) and \( \mathbf{y}_2 = A\mathbf{x}_2 \), for some vectors \( \mathbf{x}_1 \) and \( \mathbf{x}_2 \) in \( \mathbb{R}^4 \). What fact allows you to conclude that the system \( A\mathbf{x} = \mathbf{w} \) is consistent?
    **Solution:** Since \( \mathbf{w} = A(\mathbf{x}_1 + \mathbf{x}_2) \), the system is consistent due to the linearity of matrix multiplication.

    \item[36.] 
    Let \( A \) be a \( 5 \times 3 \) matrix, let \( \mathbf{y} \) be a vector in \( \mathbb{R}^3 \), and let \( \mathbf{z} \) be a vector in \( \mathbb{R}^5 \). Suppose \( A\mathbf{y} = \mathbf{z} \). What fact allows you to conclude that the system \( A\mathbf{x} = 4\mathbf{z} \) is consistent?
    **Solution:** Since \( A\mathbf{y} = \mathbf{z} \), multiplying both sides by 4 gives \( A(4\mathbf{y}) = 4\mathbf{z} \), ensuring the system is consistent with \( \mathbf{x} = 4\mathbf{y} \).
\end{enumerate}

}

\end{document}
